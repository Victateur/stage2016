\noindent -"Le Flan" (plein de gens)

\noindent -Damien : "Apprenez le Brainfuck à quelqu'un, il codera mal un jour. Apprenez-lui le java, il codera mal toujours."

\noindent -Pierre : C'est du comique pas drôle de répétition.

\noindent Damien : C'est du comique pas drôle de répétition.

\noindent -Savinien :  "VOUS N'ETES PAS PRÊTS !"

\noindent -Le conférencier : "On a $100$ prisonniers..."

\noindent Arthur : "C'est le problème des prisonniers ?"
 
\noindent -Henry : "On s'est trompés dans l'endroit où on a mis l'erreur."

\noindent -Le conférencier : "Voici Animours, la mascotte française qui a tristement fini ses jours en Thaïlande, demandez à Vincent de vous raconter l'histoire..."

\noindent Les Belges : "Il faudra vraiment qu'on leur dise un jour."

\noindent -Timothée : "La salle de bain est si petite qu'on peut prendre une douche en pissant dans les toilettes."

\noindent -Yakob : "Dépêchez-vous, on perd la course au flan !"
\\

\textit{(Durant la visite du musée)} 

\noindent -Corentin : "Ceci n'est pas de l'art, c'est un webdriver [marque quelconque] $2.0$."

\noindent Félix : "la meilleure oeuvre d'art c'était le thermostat"

\noindent Antoine : "CodeGear et Deathnote c'est pareil, on a Coca-zero et coca-light" [tapé par un animatheur qui ne voit pas le lien avec le musée, mais bon...]
\\

\noindent -Joon : "Mais ça se voit visuellement bien" [Note des élèves : ...sur la figure malfaite]

\noindent -Victor : "Soit $P,A,B,C,D$ quatre points."

\noindent -Thomas : "Le point $G$, il est à peu près on ne sait pas où..."
\\

\textit{(Au loup-garou)}

\noindent -Rémi : "On est $7$, il  reste $3$ frères, un chasseur, deux soeurs, une servante et $4$ loups-garous. LE MJ NE SE FOUTRAIT PAS UN PEU DE NOTRE GUEULE ?"

\noindent Timothée : "Le boucher me désigne Pierre-alexandre... heu, la personne dont il veut couper la langue."

\noindent Timothée \textit{(après que le village a voté la mort du montreur d'ours)} : "Suite à un souci de pilosité, les villageois ont pendu l'ours et empaillé le montreur."
\\

\textit{(Au Mao)}
\noindent -Pierre (après $2$h$30$ de jeu) : "Vous vous souvenez du début de la partie ?" 

\noindent Corentin : "Je ne me souviens plus de mon nom !"

\noindent Savinien : "Le mao est un jeu qui se joue à six paquets de cartes, un échiquier, de cartes de tempêtes sur un échiquier, une table d'inversion modulo $13$, de la musique..."

\noindent Timothée : "... et deux flans !"
\\

\noindent Anonyme : "J'ai perdu." [NdLR : on a dû l'avoir chaque année, celle-là]

\noindent -Antoine (en test) : "$26$ est congru à $0$ à modulo $4$."

\noindent -Timothée : "J'ai eu un score parfait au test, j'ai eu $6$, c'est un nombre parfait !"

\noindent Rémi : "Est-ce que $1$ est un nombre parfait ?"

\noindent -Clara : "C'est quoi le CIV ?"

\noindent -Lucie : "Musée Fernand Léger, ça envoie du lourd !"

\noindent -Adrien : "C'est du beach-volley, mais sans \textit{biatch}, quoi" [taper ceci fait pleurer un animatheur comme s'il pelait des oignons]